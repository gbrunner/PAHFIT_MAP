\documentclass[manuscript]{aastex}
\newcommand{\myemail}{gbrunner@rice.edu}

\shorttitle{Molecular Hydrogen in NGC 5194}
\shortauthors{Brunner et al.}


\begin{document}

\title{Warm Molecular Gas in M51: Mapping $\mathrm{H_2}$ Excitation
  and Mass with the Spitzer Infrared Spectrograph}

\author{Gregory Brunner\altaffilmark{1,2}}
\affil{Department of Physics and Astronomy, Rice University, \\
    Houston, TX 77005}
\email{gbrunner@ipac.caltech.edu, gbrunner@rice.edu}

\author{Kartik Sheth\altaffilmark{3}, Lee Armus\altaffilmark{3}, and George Helou\altaffilmark{3}}
\affil{Caltech, Spitzer Science Center, Pasadena, CA 91125}

\author{Eva Schinnerer\altaffilmark{4}}
\affil{Max-Planck-Intit\"{u}t f\"{u}r Astronomie, Heidelberg, Germany}

%\and 

\author{Stuart Vogel\altaffilmark{5} and Mark Wolfire\altaffilmark{5}}
\affil{Department of Astronomy, University of Maryland, College Park, MD 20741}

\and

\author{Reginald Dufour\altaffilmark{1}}
\affil{Department of Physics and Astronomy, Rice University, Houston, TX 77005}

\altaffiltext{1}{Department of Physics and Astronomy, Rice University, Houston, TX 77005}
\altaffiltext{2}{Visiting Graduate Student Fellow, Spitzer Science Center, Caltech, Pasadena, CA 91125}
\altaffiltext{3}{Spitzer Science Center, Caltech, Pasadena, CA 91125}
\altaffiltext{4}{Max-Planck-Instit\"{u}t f\"{u}r Astronomie, Heidelberg, Germany}
\altaffiltext{5}{Department of Astronomy, University of Maryland, College Park, MD 20741}


\begin{abstract}

We have mapped the molecular gas in a radial strip across the disk of
M51 using $\mathrm{H_2}$ S(0) $-$ $\mathrm{H_2}$ S(5) pure rotational
mid-infrared emission lines using the Infrared Spectrograph (IRS) on
the Spitzer Space Telescope.  The morphology of the molecular gas
changes significantly in the different line maps.  We find that the
$\mathrm{H_2}$ excitation temperature varies from 100 $-$ 600 K across
the galactic disk.  The nucleus contains the hottest gas (T$\sim$600
K) and the interarm regions have the coolest (T$\sim$??) gas.  We
measured the $\mathrm{H_2}$ mass surface density by modeling the gas
distribution with a two temperature model: a warm (T = 100 $-$ 300 K)
phase using the low $J$ lines and a hot phase (T = 400 $-$ 600 K)
using the high $J$ lines.  The hot phase gas surface density peaks at
0.24 $\mathrm{M_\sun}$/$\mathrm{pc^2}$ at the nucleus, wheras the warm
gas phase surface density reaches a maximum of xx
$\mathrm{M_\sun}$/$\mathrm{pc^2}$ in the north-eastern spiral arm.
The spatial variation in the hot and warm phases indicates that the
molecular gas is being heated by different excitation mechanisms in
the nucleus and in the spiral arms.  There are significant spatial
offsets between the warm H$_2$ phase and the colder ($\sim$ 10 K)
$\mathrm{H_2}$ gas traced by CO (J = 1 --0) emission, such that the
warm H$_2$ is offset towards the downstream side of the cold gas.  The
warm H$_2$ emission is spatially coincident with H$\alpha$ emission,
indicating that the warm H$_2$ gas phase is primarily heated by the
HII regions.  In contrast the hotter H$_2$ emission is spatially
coincident with [O IV](25.89 $\micron$) and X-ray emission, indicating
that shocks and/or X-rays are responsible for exciting this phase.


\end{abstract}

\keywords{galaxies: ISM --- galaxies: $\mathrm{H_2}$ --- galaxies:
  individual(M51)}

\section{Introduction}

Star formation and galactic evolution are connected via the molecular
gas in a galaxy.  In the Milky Way, nearly all star formation occurs
in molecular clouds \citep{bl95}, although not all molecular clouds
are actively forming stars.  On a more global, galactic scale, star
formation may be triggered whenever the molecular gas surface density
is enhanced, for example, by a spiral density wave \citep{vog88}, by
increased pressure or gas density in galactic nuclei \citep{you91,
  she05}, due to hydrodynamic shocks along the leading edge of bars
\citep{she00, she02}, and in the transition region at the ends of bars
\citep{kl91,she02}.  How does this star formation affect the
surrounding molecular gas?  How is it heated and what is the
distribution of the gas temperatures?  How does the mass of the warm
and hot gas vary from region to region?  We address these questions
using a spectral map of a radial strip across the spiral galaxy, M51.

M51 (also known as the Whirlpool galaxy and NGC 5194) is a nearby,
face-on spiral galaxy that is rich in molecular gas.  Its proximity
(assumed to be 8.2 Mpc \citep{tul88}), face-on orientation, and
grand-design spiral morphology make it the ideal target for many
studies of the interstellar medium (ISM) across distinct dynamical,
chemical, and physical environments in a galaxy.  Studies of the
molecular gas within M51 have revealed giant molecular associations
(GMAs) along the spiral arms \citep{aal99}, a reservoir of molecular
gas associated with the nuclear AGN \citep{sco98}, and star-formation
in molecular clouds triggered by a spiral density wave \citep{vog88}.
In addition to being well-studied at millimeter and radio wavelengths,
M51 has been also studied at X-ray, UV, optical, near-infrared,
infrared, and submillimeter wavelengths \citep{pal85,ter01, sco01,
  cal05, mat04}.  Here we present new mid-infrared spectra obtaineed
with the Infrared Spectrograph (IRS) \citep{hou04} on-board the
$Spitzer$ $Space$ $Telescope$.  Pure rotational emission lines of
$\mathrm{H_2}$ in the mid-infrared are powerful tools for observing
the warmer ($>$100 K) phase of $\mathrm{H_2}$ \footnote{Although we
  are always exploring the warm phase of $\mathrm{H_2}$, in this paper
  we always refer to a warm and hot phase corresponding to
  temperatures of, T = 100 $-$ 300 K, and T = 400 $-$ 1000 K
  respectively.}  The mid-infrared $\mathrm{H_2}$ lines are important
diagnostics of the ISM because they allow us to model the
$\mathrm{H_2}$ excitation-temperature, mass \citep{rig02, hig06}, and
ortho-to-para ratio \citep{neu98, neu06}.  These quantitites, in turn,
help us constraint the energy injection mechanism (i.e. radiative
heating, shocks, turbulence) that heats the warm molecular gas phase
in the ISM.

\section{Observations and Data Reduction}

\subsection{Spectral Data}

We mapped a radial strip across \objectname{M51} using the short-low
(SL) and long-low (LL) modules of the $Spitzer$ IRS in spectral
mapping mode.  The radial strips were 324$\arcsec$ $\times$
57$\arcsec$ and 295$\arcsec$ $\times$ 51$\arcsec$ in the SL and LL,
respectively.  Integration times in the SL and LL were 14.6 s.  Each
slit position was mapped twice and half-slit spacings were used.  In
total, 1,412 spectra were taken in the SL and 100 were taken in the LL
(including background observations).  Dedicated off source background
observations were taken for the SL observations.  Backgrounds for the
LL observations were taken from outrigger data collected while the
spacecraft was mapping in the adjacent module.  The astronomical
observation requests (AORs) are available on SST's Leopard and Spot
(Project ID 200138, PI: K. Sheth).

The spectra were assembled from the basic calibration data (BCD) into
spectral data cubes for each module using CUBISM \citep{smi04}.
Background subtraction and bad pixel removal was done within CUBISM.
The individual IRS spectra were processed using the S14.0 version of
the Spitzer Science Center (SSC) pipeline.  In CUBISM, the SL and LL
data cubes have 1.85$\arcsec$ pixels and 5.08$\arcsec$ pixels
respectively.  This pixel size is half the point spread function (PSF)
at the red end of the wavelength sampled by the modules.  In
principle, the PSF should vary with wavelength but since the PSF is
undersampled at the blue end of the module, it is approximately
constant across a given module.  So the approximate resolution of the
SL module is xx arcsecs and of the LL module is yy arcsecs. ** DELETE
THAT COLUMN IN TABLE 1**


%Table 1 lists the resolutions of the $\mathrm{H_2}$ line maps.

We created continuum subtracted line-only maps of the $\mathrm{H_2}$
S(0) $-$ $\mathrm{H_2}$ S(5) using a combination of PAHFIT
\citep{smi07} and our own code.  PAHFIT is a spectral fitting routine
that decomposes IRS low resolution spectra with the main advantage
being that it allows one to recover the full line flux of any blended
features.  Several $\mathrm{H_2}$ lines are blended: the
$\mathrm{H_2}$ S(1) with the 17.0 $\micron$ PAH complex,
$\mathrm{H_2}$ S(2) with the [Ne II](12.8 $\micron$), and the
$\mathrm{H_2}$ S(5) with the [Ar II](6.9 $\micron$).  We first
concatenated SL1 and SL2, and LL1 and LL2 data cubes into two cubes,
one for SL and one for LL.  We smoothed each map in the cubes by a 3
$\times$ 3 pixel box, conserving the flux, to increase the
signal-to-noise ratio of the spectra.  Then, for each pixel, we
extracted a spectrum and ran PAHFIT to decompose it.  We saved the
location of the pixel along with the PAHFIT output and reconstructed
de-blended features and line maps for all of the mid-infrared
features.  In this paper we focus only on the $\mathrm{H_2}$ line maps.

\subsection{Ancillary Data: CO (J=1--0), H$\alpha$, and X-ray Observations}

The BIMA (Berkely Illinois Maryland Array) CO (J = 1 $--$ 0) map was
acquired as a part of the BIMA Survey of Nearby Galaxies (SONG)
\citep{reg01, hel03}.  At the distance of M51, the SONG beam
(5$\farcs$8 $\times$ 5$\farcs$1) subtends 220 pc $\times$ 190 pc.  The
H$\alpha$ + [N II] image of M51 was observed at Kitt Peak as part of
$Spitzer$ $Infrared$ $Nearby$ $Galaxies$ $Survey$ (SINGS).  The native
pixel scale for that image is 0$\farcs$3. Additional information about
the H$\alpha$ image can be found in the SINGS Fourth Data Release
(DR4).  X-ray emission from M51 was observed by the Advanced CCD
Imaging Spectrometer (ACIS) on the $Chandra$ $X$-$Ray$ $Observatory$
on 20 June 2000.  The total integration time was 14,865 seconds.  The
resolution of the image is $\sim$ 1$\arcsec$.  Further details of the
observations are presented in \citep{wil01}.

\section{Results}

\subsection{Morphology of $\mathrm{H_2}$ Emission}

*** MUST PUT A CROSS FOR THE NUCLEAR POSITION.  ALSO CONVERT ALL JPGS TO
*** EPS FOR SUBMISSION.  KEEP THE JPGS THOUGH FOR TALKS / ETC.  ALSO RENAME 
*** ALL FIGS to FIG1, FIG2 BECAUSE THIS IS THE FORMAT REQUESTED BY APJ.

We have detected and mapped $\mathrm{H_2}$ emission from the six
lowest pure rotational $\mathrm{H_2}$ lines (Figure \ref{fig1}).  The
maps reveal remarkable differences in the distribution of the
$\mathrm{H_2}$ emission in M51.  $\mathrm{H_2}$ S(0) emission is
strongest in the northwest spiral arm peaking at an intensity of 3.66
$\times$ $\mathrm{10^{-18}}$ $\mathrm{W/m^2}$ and decreases by a
factor of 2 in the nuclear region.  In contrast, the $\mathrm{H_2}$
S(1) emission peaks in the nucleus of the galaxy at an intensity of
1.03 $\times$ $\mathrm{10^{-17}}$ $\mathrm{W/m^2}$ and has an
extension of equal intensity towards the northwest spiral arm.  In the
spiral arm itself, the $\mathrm{H_2}$ S(0) peak is offset from the
$\mathrm{H_2}$ S(1) emission by 10$\farcs$2 ($\sim$380 pc).  These
offsets cannot be explained by the difference in the resolution
between the SL and LL map ** CHECK THIS BY SMOOTHING THE SL MAP TO LL
RESOLUTION AND VERIFYING THAT THE TWO PEAKS ARE STILL DISTINCT. ** We
find $\mathrm{H_2}$ S(0) and $\mathrm{H_2}$ S(1) emission as far as 5
$-$ 6 kpc from the nucleus of the galaxy.  In the outer spiral arm,
the $\mathrm{H_2}$ S(0) intensity is a factor of 2 times lower than in
the inner northwest spiral arm and the $\mathrm{H_2}$ S(1) intensity
is a factor of 5 times lower than in the nucleus.

** I SEE THAT YOU HAVE SEVERAL VERSIONS OF FIGURES AND I WOULD USE THE
** ONES WITH THE NUCLEAR CROSS AND LABEL ALL THE H2Sx FIGS WITH THE
** SAME LABELS
 
The $\mathrm{ H_2}$ S(2) $-$ $\mathrm{H_2}$ S(5) maps show different
molecular gas morphology within \objectname{M51} through each
$\mathrm{H_2}$ line.  The strongest $\mathrm{H_2}$ S(2) emission is
from the nucleus at 2.21 $\times$ $\mathrm{10^{-18}}$
$\mathrm{W/m^2}$.  We also see bright $\mathrm{H_2}$ S(2) emission
from the inner northwest spiral arm at half the intensity of the
nuclear peak.  The $\mathrm{H_2}$ S(3) peak at the nucleus is 1.35
$\times$ $\mathrm{10^{-17}}$ $\mathrm{W/m^2}$, a factor of $\sim$ 6
greater than the $\mathrm{H_2}$ S(2) nuclear peak.  There is also a
linear bar-like structure in $\mathrm{H_2}$ S(3) emission across the
nucleus of the galaxy at a PA$\sim$-10$^0$.  The emission peaks in the
$\mathrm{H_2}$ S(2) and $\mathrm{H_2}$ S(3) maps are not spatially
coincident.  For instance, in the inner spiral arm there is a
$\mathrm{H_2}$ S(2) peak coincident with the CO peak whereas the
$\mathrm{H_2}$ S(3)peaks further down the spiral arm.  Offsets like
these suggest that there may be variations in the excitation
temperature from region to region within a galaxy, and even within a
spiral arm.

The $\mathrm{H_2}$ S(4) and $\mathrm{H_2}$ S(5) lines are brightest at
the nucleus with intensities of 3.05 and 8.04 $\times$
$\mathrm{10^{-18}}$ $\mathrm{W/m^2}$ respectively.  The $\mathrm{H_2}$
S(4) line shows emission in the nucleus and in the spiral arm to the
west.  In the spiral arm to the west of the nucleus, the
$\mathrm{H_2}$ S(4) intensity is 2.11 $\times$ $\mathrm{10^{-18}}$
$\mathrm{W/m^2}$.  This is notable because the spiral arm to the
southwest of the nucleus is very bright in CO and and studies have
revealed very high molecular gas column densities in the southwest
inner spiral arm \citep{lor90}.  $\mathrm{H_2}$ S(5) emission is
asymmetric in the nucleus and mimics the morphology of the
$\mathrm{H_2}$ S(3) line with extended emission to the north of the
nucleus.  The differences in the morphology of $\mathrm{H_2}$ emission
are indicative of changes in the $\mathrm{H_2}$ excitation-temperature
across the galaxy, which we discuss in the next section.

\subsubsection{Note on Uncertainties in the $\mathrm{H_2}$ Line Intensity Maps}

** THIS SUBSECTION IS UNNECESSARY - WE DON''T HAVE A GOOD HANDLE ON
** IRS UNCERTAINTIES ANYWAYS AND THERE ARE SYSTEMATICS THAT ARE NOT
** REALLY TAKEN INTO ACCOUNT HERE SO I RECOMMEND DELETING IT ENTIRELY
** 

The uncertainty in the $\mathrm{H_2}$ line intensity varies as a
function of line intensity across each map.  The uncertainty in the
$\mathrm{H_2}$ S(0) line intensity ranges from 20 $-$ 40 \% across the
nuclear region and spiral arms.  The uncertainty in the spiral arm
intensity at greater distances from the center of the strip is 80 $-$
90 \%.  The $\mathrm{H_2}$ S(1) line map has lower uncertainties with
the uncertainty being $\sim$ 10 \% in the nuclear region and 20 $-$ 30
\% in the spiral arms.  The $\mathrm{H_2}$ S(2) and $\mathrm{H_2}$
S(3) lines show similar uncertainties to each other.  In the nuclear
region their uncertainty is 10 $-$ 30 \% and in the spiral arms their
uncertainty is 50 $-$ 80 \%.  The $\mathrm{H_2}$ S(4) and
$\mathrm{H_2}$ S(5) lines only show emission from the nuclear region;
their uncertainties in the line intensity are $\sim$ 35 \% and 25 \%,
respectively.

\subsection{Mapping $\mathrm{H_2}$ Excitation-Temperature and Mass across M51}

\subsubsection{Modeling $\mathrm{H_2}$ Excitation-Temperature and Mass}

The pure rotational lines of molecular hydrogen provide a powerful
probe of the conditions of the ISM by placing constraints on the
energy injection that excites $\mathrm{H_2}$ \citep{neu06}.  Following
\citet{rig02, hig06} we modeled the $\mathrm{H_2}$ excitation and mass
across M51.

First, we smoothed the $\mathrm{H_2}$ S(1) $-$ $\mathrm{H_2}$ S(5)
maps were smoothed to the resolution of the $\mathrm{H_2}$ S(0) map.
The maps were then interpolated to the same spatial grid.  Excitation
diagrams across the strip were derived from the Boltzman equation
using the formulation of Rigopoulou et al. (2002),
\begin{equation}
N_i/N = (g(i)/Z(\mathrm{T_{ex}}))exp(-T_i/\mathrm{T_{ex}})
\end{equation}

where $g(i)$ is the statistical weight of state $i$,
Z($\mathrm{T_{ex}}$) is the partition function, $\mathrm{T_i}$ is the
energy level of a given state, and $\mathrm{T_{ex}}$ is the excitation
temperature.  N and $\mathrm{N_i}$ are the total column density and
the column density of a given state $i$ and $\mathrm{N_i}$ is
determined directly from the measured extinction-corrected flux by

\begin{equation}
N_i = 4 \pi \times flux(i)/(\Omega A(i)h\nu (i))
\end{equation}

where A($i$) is the Einstein $A$-coefficient, $\nu$($i$) is the
frequency of state $i$, $\Omega$ is the solid angle of the beam, and
$h$ is Planck's constant.  Table 1 lists the values for the
wavelength, rotational state, Einstein $A$-coefficient, energy, and
statistical weight of the pure rotational levels of $\mathrm{H_2}$.

** PUT THE NAME OF THE REGION ON THE PLOTS.  ALSO PUT THE NUMBERS OF
** THE TRANSITIONS ON ONE OF THEM.  
** ALSO HOW DO I KNOW FROM THESE PLOTS THAT OPR = 3?  YOU NEED TO SAY
** HERE THAT IF OPR =3 YOU EXPECT THE FOLLOWING IN THE BOLTZMANN
** DIAGRAM AND IF NOT THEN YOU EXPECT SUCH AND SUCH...

** OVERALL THIS PARA IS VERY CONFUSING.  AT ONE POINT OPR IS 3 BUT
** SOMEHOW THIS ALSO SHOWS A RANGE OF TEMPERATURES?  VERY CONFUSED.
** THIS NEXT PARAGRAPH REALLY NEEDS TO BE REWRITTEN.

In Figure \ref{fig2} we present excitation diagrams from three
different regions across the M51 strip.  The excitation diagrams
exhibit an ortho-to-para ratio (OPR) of 3 in the nuclear region.  This
is in agreement with the value of the OPR of the nuclear region
determined by SINGS \citep{rou07}.  Outside the nucleus, the lower $J$
($\mathrm{H_2}$ S(0) - $\mathrm{H_2}$ S(3)) levels exhibit an OPR of
3.  The $\mathrm{H_2}$ S(4) measurement shows significant scatter in
the excitation diagrams outside of the nuclear region of M51.  This
would indicate that the OPR is less than 3; however, due to the low
signal-to-noise ratio of the $\mathrm{H_2}$ S(4) map, we do not
believe that the OPR determined from the $\mathrm{H_2}$ S(4) intensity
reflects the OPR of the warm $\mathrm{H_2}$.  The excitation-diagrams
also exhibit a change in slope as a function of rotational level.
This indicates that a continuous distribution of $\mathrm{H_2}$
temperatures is being sampled within the beam.


** REWRITE PARA ABOVE 

** DO YOU ASSUME TEMPERATURES FIRST AND DRAW THE LINE OR GET A
** TEMPERATURE FROM THE LINES THEMSELVES?  THAT WILL DETERMINE HOW THE
** NEXT PARA SHOULD START.  HERE I AM ASSUMING YOU GET A BEST FIT TO
** THE HIGH J LINES AND CALL THAT YOUR TEMPERATURE.  THEN SUBTRACT
** THAT CONTRIBUTION TO LOW J LINES AND THEN FIT A TEMP TO THOSE?
** CORRECT?

The simplest model beyond a single temperature model is a two
temperature model, which we call a warm and hot phase from here on.
To determine the hot phase temperature we do a least squares fit AT
EVERY PIXEL? to the $\mathrm{H_2}$ S(2) $-$ $\mathrm{H_2}$ S(5) lines.
We subtract the contribution of this hot phase temperature from the
lower J lines and do a least squares fit to the $\mathrm{H_2}$ S(0) --
$\mathrm{H_2}$ S(2) lines.  ** HOW DO YOU THEN GET THE MASS? **

\subsubsection{Warm and Hot $\mathrm{H_2}$ Mass Distributions}

In Figure \ref{fig3} we present the warm ($left$) and hot ($right$)
$\mathrm{H_2}$ mass distributions across the M51 strip\footnote{Note
  that the non-rectangular shape of the strip is due to the offset
  between the SL and LL strips.}  ** NO MAKE THE SIZES THE SAME AND
DELETE THIS SENTENCE -- Also note that the sizes of the plots of the
warm and hot $\mathrm{H_2}$ mass phases are different.  ** 

** ITS BECOMING PAINFUL TO WRITE \MATHRM EVERYWHERE.. ARE YOU SURE YOU
** NEED IT?  WHY IS H$_2$ NOT SUFFICIENT - -SAME GOES FOR EVERY TIME
** YOU USE MATHRM ?? AND PLEASE REDEFINE COMMANDS SO YOU CAN USE A
** SHORTHAND NOTATION FOR COMMONLY USED LABELS., E.G. 
** \newcommand{\kms}{km\ s$^{-1}$}  
** \newcommand{\vsys}{v_{\rm sys}}
** AT THE BEGINNING OF THE DOCUMENT. 

The highest gas surface density for the warm $\mathrm{H_2}$ phase is
in the inner northwest spiral arm at 11
** CHANGE ALL PC^2 TO PC^{-2} AND REMOVE DIVISION SIGNS
$\mathrm{M_\sun}$ $\mathrm{pc^{-2}}$ warm $\mathrm{H_2}$. The gas surface
density in the outer northwest and southeast spiral arms is maximum at
the center of the spiral arms at 3.5 $\mathrm{M_\sun}$/$\mathrm{pc^2}$
and 1.0 $\mathrm{M_\sun}$/$\mathrm{pc^2}$ respectively.
** PEAKS CANNOT BE BIMODAL BY DEFINITON - I REWORDED THAT SENTENCE!
The hot phase surface density is highest in the nucleus and interior
to the inner spiral arm at 0.24 $\mathrm{M_\sun}$/$\mathrm{pc^2}$.
The gas surface density of the hot phase in the spiral arms is
1/3--1/5th that of the nuclear region.

\subsubsection{Warm and Hot $\mathrm{H_2}$ Excitation-Temperature Distributions}

** DO YOU MEAN MASS OR SURFACE DENSITY -- PLEASE GET THE NOMENCLATURE
** CORRECT BASED ON THE UNITS YOU ARE USING.  IF PER PC^2 IT SHOULD BE
** SURFACE DENSITY.


** THIS SECTION NEEDS TO BE REWRITTEN.  WRITE IT AS FOLLOWS.  FIG XX
** AND YY SHOW IN GREYSCALE THE DISTRIBUTION OF TEMPERATURES
** DETERMINED FROM THE HIGH AND LOW TRANSITIONS AS DESCRIBED ABOVE.
** OVERLAID ON IT ARE CONTOURS OF THE H2 GAS SURFACE DENSITY.  IN BOTH
** CASES WE FIND THAT THE TEMPERATURE AND GAS SURFACE DENSITY ARE
** INVERSELY CORRELATED. -- THAT'S ALL YOU YOU NEED TO WRITE.  YOU
** NEED TO REALLY UNDERSTAND WHY THE INTERARM REGIONS HAVE SUCH HIGH
** TEMPS ACCORDING TO THIS FIGURE.  IT REALLY CANNOT BE REAL BECAUSE
** OUR S/N IS LOWEST IN THESE REGIONS.

In Figure \ref{fig4} we compare the warm $\mathrm{H_2}$ mass
distribution to the excitation-temperature.  The gas is warmest in the
nuclear region of \objectname{M51} with the temperature ranging from
175 $-$ 190 K in the center of the galaxy.  The gas is cooler within
the spiral arms of \objectname{M51} than in the inter-arm regions.
The temperature in this spiral arm decreases from 175 K at the outer
surface layers to 154 K in the most dense region of the arm indicating
that the warm $\mathrm{H_2}$ excitation-temperature is cooler at
higher $\mathrm{H_2}$ densities.  In the outer northwest and southeast
spiral arms (at 5 $-$ 6 kpc from the center of the galaxy each)
molecular the gas excitation-temperature ranges from 150 K at the
center to about 175 K near the edge of the spiral arms; again
indicating that high density regions are cooler than low density
regions.

The excitation-temperature distribution of the hot $\mathrm{H_2}$ is
shown in Figure \ref{fig5} and reveals a different picture than the
excitation-temperature distribution of the warm $\mathrm{H_2}$.  The
excitation-temperature in the nuclear region ranges from 587 to 612 K.
The excitation-temperature in the northwest inner spiral arm is the
coolest at 538 K.  The hottest excitation-temperatures are observed in
the inter-arm regions where there is the least amount of
$\mathrm{H_2}$.  

** I DON'T REALLY BELIEVE THIS --> DO YOU?  YOU MAY BE MISLED BY THE
** LOW S/N IN THESE MAPS IN THE INTERARM REGION ESP. FOR THE HIGHEST J
** LINES.
In contrast, in the southeast inter-arm region, the
excitation temperature is in excess of 900 K.

\section{Discussion}

\subsection{Comparison to Previous Studies of Warm and Hot $\mathrm{H_2}$ in Galaxies}

Previous studies have used aperture-averages over entire galactic
nuclei to derive the physical conditions of the molecular gas
\citep{rig02, hig06, rou07}.  In M51, \citet{rou07} find that within
the central 330 $\mathrm{arcsec^2}$ (ADD PHYSICAL SCALE IN XX PC$^2$),
the warm $\mathrm{H_2}$ phase has a maximum temperature of 180 K and a
total mass of $\mathrm{M_{warm}}$ = 1.46 $\times$ $\mathrm{10^6}$
$\mathrm{M_\sun}$.  In the central 412 $\mathrm{arcsec^2}$ of M51, we
find a corresponding maximum of 186 K and a total mass of 1.63
$\times$ $\mathrm{10^6}$ $\mathrm{M_\sun}$, consistent with the SINGS
results.  The SINGS team also measured the excitation-temperature of
the hot phase (though they do not measure the mass in the hot phase)
and find a hot $\mathrm{H_2}$ excitation-temperature of 521 K.  Over
the same region, we find a maximum excitation-temperature of 584 K.

** BUT YOUR REAL STRENGTH IS THE SPATIAL RESOLUTION AND YOU SHOULD
** REALLY REWRITE THE TOP PARAGRAPH TO REFLECT THAT.  SO SAY THAT WE
** HAVE THE ABILITY TO DO SPATIALLY RESOLVED STUDIES BUT IT WOULD BE
** GOOD TO COMPARE OUR NUMBERS TO THE AVERAGES FOUND BY PREVIOUS
** STUDIES.  THEN SAY THAT THE SPATIALLY RESOLVED NUMBERS MATCH THE
** AVERAGED STUDIES SO MUCH SO THAT THE AVERAGED STUDIES MUST BE
** DOMINATED BY THE BRIGHTEST REGIONS.. OR IT COULD BE THAT YOU ARE
** AVERAGING THE NUMBERS YOU QUOTE ABOVE -- IS THAT THE CASE?  I AM
** CONFUSED A BIT HERE.

** I DON'T SEE THE POINT OF THE PARA YOU HAD HERE BELOW.  I WOULD DELETE IT.  
In a survey of Seyfert galaxies, Rigopoulou et al. (2002) find that the
warm-to-hot phase mass fraction varies between 5 $\times$
$\mathrm{10^3}$ $-$ 1 $\times$ $\mathrm{10^6}$.  This is significantly
higher than the warm-to-hot mass fraction (Figure \ref{fig6}) than we
measure within the central 412 $\mathrm{arcsec^2}$ of M51, 14.8.  To
measure the hot $\mathrm{H_2}$ mass, we use the $\mathrm{H_2}$ S(2)
$-$ $\mathrm{H_2}$ S(5) lines, whereas Rigopoulou et al. (2002) used
the $\mathrm{H_2}$ S(5) $-$ $\mathrm{H_2}$ S(7) lines.  Thus, the hot
$\mathrm{H_2}$ phase that they measured is preferentially offset to
higher $\mathrm{H_2}$ excitation-temperatures and lower $\mathrm{H_2}$
masses.

\subsection{The Warm-to-Hot $\mathrm{H_2}$ Mass Ratio}

** NOW WE GO BACK TO DESCRIBING WARM VS. HOT?  I THOUGHT WE WERE IN
** THE DISCUSSION SECRION HERE. THIS IS REPETITIOUS.  DIDN'T YOU
** ALREADY DO THIS BEFORE?  JUST REDUCE THE FOLLOWING PARA TO ONE
** LINE.  SAY THAT THE WARM TO HOT RATIO VARIES (PERHAPS YOU CAN SHOW
** THE DIVISION OF THE TWO MAPS RATHER THAN OVERLAYING THEM WHICH DOES
** NOT CONVEY QUANTITATIVE INFO.  YOU JUST NEED TO SAY THIS IS THE
** RATIO SEE...

In Figure \ref{fig6}, we compare the warm $\mathrm{H_2}$ mass
distribution to the hot $\mathrm{H_2}$ mass distribution.  The warm
$\mathrm{H_2}$ mass distribution peaks at in the northwest inner
spiral arm and the hot $\mathrm{H_2}$ mass distribution peaks at in
the nucleus and in the region interior to the northwest spiral arm.
It is evident from the figure that the warm-to-hot $\mathrm{H_2}$ mass
ratio is not constant across the galaxy but is lowest in the nucleus
of the galaxy (at 12) and increases to 170 and 136 in the southeast
and northwest spiral arms, respectively.  The differences between the
warm and hot $\mathrm{H_2}$ mass distributions and the variations in
the morphology of M51 as a function of rotational energy level suggest
different origins to the warm and hot $\mathrm{H_2}$ phases and that
the primary excitation mechanisms of the warm and hot $\mathrm{H_2}$
phases differ.
 
** THEN IN THIS PARA JUST SAY - THE MOST OBVIOUS HEATING MECHANISM FOR
** THE MOLECULAR GAS IS STAR FORMATION OR NUCLEAR ACTIVITY. THE FORMER
** CAN BE TRACED USING HALPHA AND HTE LATTER USING XRAYS AND OIV LINE.
** IT WOULD ALSO BE INTERESTING TO SEE HOW THE WARM GAS WAS
** DISTRIBUTED RELATIVE TO THE COLD GAS - SO WE ALSO COMPARE OUR DATA
** TO CO DATA FROM SONG.

In order to understand the $\mathrm{H_2}$ excitation and where each
excitation mechanism is dominant, we made comparisons of the
$\mathrm{H_2}$ line intensity and mass distributions to diagnostics of
three excitation mechanisms: UV photons emitted in dense photon
dominated regions (PDRs) and H II regions, shocks, and X-rays.  We use
CO (J = 1 $-$ 0) emission to determine the location of the
$\mathrm{H_2}$ relative to locations dense PDRs, H$\alpha$ imagery to
identify H II regions \citep{sco01}, [O IV](25.98 $\micron$) line
emission as a diagnostic of shocks \citep{ss99}, and the 0.5 $-$ 10.0
keV X-ray band to distinguish X-ray dominated regions (XDRs).  The
following section compares the $\mathrm{H_2}$ mass distributions to CO
(J = 1 - 0), H$\alpha$, [O IV], and X-ray emission.

\subsection{Distinguishing the $\mathrm{H_2}$ Excitation Mechanisms}

** AS A GUIDE TO DECREASING THE VERBOSITY OF THE ARTICLE, TRY AND
** REDUCE THE NEXT THREE SUBSECTIONS TO THREE PARAGRAPHS OF NO MORE
** THAN 10 SENTENCES (AND NO CHEATING WITH RUN-ON SENTENCES).  THIS
** WILL MAKE THIS PAPER MUCH MORE COMPACT AND TRACTABLE. I WILL TAKE
** OUT THE SALIENT SENTENCES FROM THE FIRST SUBSECTION HERE AND PUT
** THEM INSIDE *** - THESE OUGHT TO BE THE 10 SENTENCES OR LESS YOU
** FOCUS ON.

\subsubsection{$\mathrm{H_2}$ Excitation by UV Photons from PDRs: Comparison of $\mathrm{H_2}$ to CO Emission}

In star-forming regions, $\mathrm{H_2}$ exists within the PDR and deep
into the molecular cloud.  ** Owing to a low dissociation energy of 4.5
eV, $\mathrm{H_2}$ formation generally does not occur within a PDR
until the radiation field becomes sufficiently weak.  ** Understanding
$\mathrm{H_2}$ in PDRs and implementing $\mathrm{H_2}$ into
photoionization codes has become the emphasis of many theoretical
models.  Recent advances in the CLOUDY photoionization code have
included $\mathrm{H_2}$ and modeled the structure of star-forming
regions while treating the H II region and PDR as one continuous cloud
\citep{shaw05, abel05}.  Kaufman et al. (2006) used
Starburst99/Mappings to model $\mathrm{H_2}$ pure rotational line
emission from PDRs to probe the conditions of dense PDRs in
star-forming regions.  ** They show that within galaxies, where the
telescope beam size is generally kiloparsecs across, $\mathrm{H_2}$
emission could serve to probe the surface layers of dense molecular
clouds. **

** HERE ARE YOU SAYING THAT THERE IS NO H2 WHERE THERE IS CO??  THAT IS NOT RIGHT. PERHAPS YOU MEAN TO SAY CO TRACES THE COLD GAS WHERE NO MIDIR H2 EMISSION IS SEEN -- BUT THERE IS H2 THERE? RIGHT? **
CO is found deep within a molecular cloud where the temperatures are
too cold to excite $\mathrm{H_2}$ emission.  In these regions, CO is
collisionally excited by the more abundant $\mathrm{H_2}$ molecule.
CO is connected to $\mathrm{H_2}$ in star-forming regions because at
the surface layers of the molecular clouds, $\mathrm{H_2}$ is excited
and CO emits dipole rotational lines due to heating by the ionizing
radiation of newborn massive stars \citep{all04}.

In Figure \ref{fig7}, we compare the warm ($left$) and hot
$\mathrm{H_2}$ ($right$) mass distributions to CO (J = 1 $-$ 0)
emission.  The brightest CO emission is seen in the spiral arms where
the greatest amounts of warm $\mathrm{H_2}$ mass are also found.  ** The
most striking result is that in the inner spiral arms, we see that the
CO is offset toward the nucleus from the warm $\mathrm{H_2}$ mass.
The offset between the peaks in CO and warm $\mathrm{H_2}$ mass is
7$\farcs$2 in the northwest spiral arms and 5$\arcsec$ in the
southeast spiral arms.  We believe that these offsets are real with
one possible explanation being that the $\mathrm{H_2}$ is tracing the
regions of active star-formation within the giant molecular
associations. ** REDUCE ABOVE SENTENCES INSIDE ** I IN THIS PARA TO 2 SENTENCES

We compare CO to the hot $\mathrm{H_2}$ mass distribution (Figure
\ref{fig7}, $right$) and we see that the hot $\mathrm{H_2}$ mass is
most abundant in the nuclear region, interior to the CO bright spiral
arms.  In northwest spiral arm, the hot $\mathrm{H_2}$ mass is offset
from the CO by 2$\farcs$6. We believe that the offsets between the CO
and the hot $\mathrm{H_2}$ mass are real, however, they are likely due
to the dominance of other excitation mechanisms (shocks or X-rays) in
the nuclear region of M51.

*** THIS PARA REALLY CONFUSED ME.  WHAT IS BELIEVABLE AND WHAT IS NOT.
*** WHAT IS THE CO DEPTH IS NOT SUFFICIENT.  LET'S TALK ABOUT THIS
*** OVER THE PHONE OR ICHAT

In Figure \ref{fig8}, we compare the CO intensity to the
$\mathrm{H_2}$ S(0) $-$ $\mathrm{H_2}$ S(3) line intensity maps.  We
see that the $\mathrm{H_2}$ S(0) contours trace the CO spiral arms
with one notable difference being that we detect $\mathrm{H_2}$ S(0)
emission far (5 $-$ 6 kpc) from the nucleus of M51, where there is no
CO detected.  The $\mathrm{H_2}$ S(1) $-$ $\mathrm{H_2}$ S(3) contours
also trace the CO spiral arms; however, within the spiral arms, the
$\mathrm{H_2}$ is not necessarily aligned with the CO.  For example,
in the comparison of $\mathrm{H_2}$ S(3) to CO, we see that in the
inner spiral arms, the $\mathrm{H_2}$ contours trace the CO; however,
within the northwest arm, we see that there is strong $\mathrm{H_2}$
emission offset to the west by 9$\farcs$3 from the bright CO emission
in the same spiral arm.  The offsets between the peaks in
$\mathrm{H_2}$ and CO emission are not systematic in any direction.

** The CO intensity in the nucleus of the galaxy is much fainter than
in the bright CO spiral arms.  ** JUST SAYS CO -> COLD GAS AND NO COLD
GAS IN NUCLEUS. **  This is interesting because $\mathrm{H_2}$ S(1) $-$
$\mathrm{H_2}$ S(3) emission is brightest in the nucleus of M51 where
$\mathrm{H_2}$ emission from the higher $J$ lines is likely due to
more energetic processes such as shock excitation and X-rays (which we
discuss in \S4.3.3 and \S 4.3.4).

\subsubsection{$\mathrm{H_2}$ Excitation by UV Photons from H II Regions: Comparison of $\mathrm{H_2}$ to H$\alpha$ Emission}

** SENTENCES LIKE THE FIRST FOUR BELOW ARE UNNECESSARY IN A PAPER!
** OKAY IN A THESIS BUT STILL MAKES A PAPER WAAAY TO VERBOSE.  JUST
** STICK TO THE RELEVANT, IMPORTANT STUFF THAT IS ADDING TO THE
** SCIENCE.  ONLY GIVE EXTRA INFO IF ABSOLUTELY NEEDED TO SET UP AN
** ARGUMENT.

** AGAIN REDUCE THIS SUBSECTION TO 10 SENTENCES AND 1 PARA ONLY.
** BOTTOM LINE IS YOU ARE SAYING WARM H2 NOT CORRELATED WITH HALPHA
** RIGHT?  WHAT IS THE HALPHA WAS EXTINCTED?  WHAT ABOUT PASCHEN ALPHA
** - IS THAT A BETTER CORRELATION?

H II regions are
sites of recent massive star formation.  
They illuminate bright nebulae in distant galaxies and outline the
spiral arms.  H II regions emit prodigious amounts of UV radiation at
energies $\ge$ 13.6 eV capable of exciting and ionizing
$\mathrm{H_2}$.  H$\alpha$ imagery is often used to identify and map H
II regions in star-forming galaxies.  Scoville et al. (2001) used
H$\alpha$ and Pa$\alpha$ imagery to identify and characterize over
1,350 H II regions in M51. ** 

In Figure \ref{fig9}, we compare the warm ($left$) and hot ($right$)
$\mathrm{H_2}$ mass distributions to H$\alpha$ emission.  In general,
the warm and hot $\mathrm{H_2}$ concentrations are not cospatial with
the H$\alpha$ emission regions in the spiral arms with the one
exception being that the warm $\mathrm{H_2}$ mass in the inner spiral
arms appears to trace the H$\alpha$ emission.  The warm $\mathrm{H_2}$
mass contours show that local peaks in $\mathrm{H_2}$ mass are found
within the dust lanes.  An example of this is in the northwest spiral
arms where we see the $\mathrm{H_2}$ mass offset from the H$\alpha$
spiral arms with local peaks being found in the dust lanes.

In Figure \ref{fig10}, we compare the $\mathrm{H_2}$ S(0) $-$
$\mathrm{H_2}$ S(3) line intensity maps to H$\alpha$ emission.
Comparison of the $\mathrm{H_2}$ S(0) map to H$\alpha$ reveals that
the strongest $\mathrm{H_2}$ emission in the northwest and southeast
inner spiral arms is coincident with H$\alpha$ emission; however, the
other $\mathrm{H_2}$ S(0) spiral arms show the strongest emission in
the dust lanes, offset from the H$\alpha$ spiral arms.  The largest
offsets are seen in the southwest spiral arm where the $\mathrm{H_2}$
S(0) emission is offset from the H$\alpha$ spiral arm by $\sim$
15$\arcsec$ (560 pc).  $\mathrm{H_2}$ S(1) emission appears to follow
the dust lanes and the $\mathrm{H_2}$ S(1) intensity subsides into the
H$\alpha$ spiral arms.  $\mathrm{H_2}$ S(2) and $\mathrm{H_2}$ S(3)
emission is also found in the dust lanes; however, there are instances
(such as in the southeast spiral arm) where the $\mathrm{H_2}$
emission appear to be found straddling the dust lane and H$\alpha$
spiral arm.

These results are in contradiction to a previous study of
rovibrational $\mathrm{H_2}$ line emission in Seyfert galaxies by
Quillen et al. (1999) who used $HST$ NICMOS to map the
$\mathrm{H_2}$(1-0)S(1) (2.121 $\micron$) and
$\mathrm{H_2}$(1-0)S(3)(1.957 $\micron$) lines in 10 Seyfert galaxies.
They found that the $\mathrm{H_2}$ emission is generally coincident
with H$\alpha$ emission and near the dust lanes (though offset from
them).  One explanation for this could be that the rovibrational lines
trace a much hotter (1000 $-$ 5000 K) molecular gas which can be found
associated with H II regions whereas the cooler $\mathrm{H_2}$ traced
by the lowest pure rotational transitions is found in dense PDRs and
molecular clouds associated with star forming regions.

\subsubsection{$\mathrm{H_2}$ Excitation by Shocks: Comparison of $\mathrm{H_2}$ to [O IV](25.89 \micron) Emission}

** THIS SUBSECTION IS ACTUALLY WELL WRITTEN AND GETS TO THE POINT
** IMMEDIATELY - QUITE IN CONTRAST TO THE PREVIOUS TWO.  SO TRY TO
** FOLLOW THIS MODEL.  BUT IT IS STILL TOO LONG AND VERBOSE SO CUT IT
** DOWN TO ONE PARA IF POSSIBLE.

The [O IV](25.89 \micron) line can be excited in shocks \citep{ss99},
the stellar winds of massive Wolf-Rayet stars \citep{lutz98}, or by an
active galactic nucleus (AGN)\citep{smi04}.  Though the [O IV] line is
blended with the [Fe II](25.99 \micron) line in $Spitzer$ IRS low
resolution spectra, PAHFIT can deblend the two lines and in mapping
the $\mathrm{H_2}$ S(0) and $\mathrm{H_2}$ S(1) line in the LL data
cubes, we also mapped the [O IV] line.
%emission was also be mapped at a resolution of 9$\farcs$39 ($\lambda$/$\delta$$\lambda$ = 142). 

In Figure \ref{fig11}, we compare the [O IV] intensity to the warm
($left$) and hot ($right$) $\mathrm{H_2}$ disributions.  The [O IV]
emission is brightest in the nuclear region at 8.75 $\times$
$\mathrm{10^{-18}}$ $\mathrm{W/m^2}$ and the peak is coincident with
the nuclear peak in the mass of the hot $\mathrm{H_2}$.  [O IV]
emission subsides from the nucleus to the inner spiral arm by 50 \%.
We resolve weaker [O IV] emission within the warm and hot
$\mathrm{H_2}$ spiral arms.  The [O IV] intensity in the spiral arms
is a factor of $\sim$ 6 lower in the spiral arms than the peak
intensity found in the nucleus.

The [O IV] emission within the nuclear region of M51 is likely due to
the weak Seyfert II nucleus \citep{ford85} and is possibly associated
with shocked gas from the outflows of the AGN.  The peak of the [O IV]
emission coincides with the nuclear peak in hot $\mathrm{H_2}$ mass
indicating that the hot $\mathrm{H_2}$ phase in the nuclear region of
the galaxy is AGN or shock heated.  With 2.43 $\times$ $\mathrm{10^5}$
$\mathrm{M_\sun}$ of hot $\mathrm{H_2}$ in the central 0.58
$\mathrm{kpc^2}$, it is unlikely that the hot $\mathrm{H_2}$ is
fueling the central AGN, but is excited by the AGN or shocks produced
by it.  In the nuclear region we observe a factor of 12 times greater
warm $\mathrm{H_2}$ mass.  The warm $\mathrm{H_2}$ mass is much
greater within the spiral arms than within the nucleus and the
warm-to-hot mass ratio is lowest in the nuclear region where the [O
  IV] intensity is greatest.  In the nuclear region, shocks appear to
be a more efficient means to excite the hot $\mathrm{H_2}$ phase than
the warm $\mathrm{H_2}$ phase.

% the warm $\mathrm{H_2}$ is not primarily excited by shocks.

\subsubsection{$\mathrm{H_2}$ Excitation by X-rays: Comparison of $\mathrm{H_2}$ to X-ray Emission}

M51 has been extensively studied in X-rays by ASCA \citep{ter98},
Newton XMM \citep{dew05}, and the Chandra X-ray Observatory
\citep{wil01}. These observations of M51 have revealed more than 80
X-ray sources.  Candidates for these X-ray sources include neutron
stars, black holes, supernova remnants (SN 1994I), and a
low-luminosity AGN \citep{imm02, wil01}.  X-ray emission from the
nuclear region of M51 has been studied by Terashima and Wilson (2001).
They observe X-ray emission from the nucleus, the extranuclear cloud
(XNC, to the south of the nucleus), and the northern loop.  A radio
jet has been observed connecting the nucleus of M51 to the XNC in 6 cm
imagery \citep{cra92}.  The jet emanates from the south of the
elongated nucleus and is shock heating ISM.

In Figure \ref{fig12}, we compare the smoothed 0.5 - 10 keV band X-ray
image to the warm ($left$) and hot ($right$) $\mathrm{H_2}$ mass
distributions.  The 0.5 - 10 keV band has been smoothed to the
resolution of the warm and hot $\mathrm{H_2}$ mass distributions and
the nucleus, XNC, and northern loop are indistinguishable in the
smoothed image.  X-ray emission is most intense from the nucleus and
decreases into the northwest spiral arm that contains the greatest
$\mathrm{H_2}$ mass.  There appears to be very little connection
between the 0.5 - 10 keV X-ray band and the warm $\mathrm{H_2}$ mass
distribution.

The peak in X-ray emission is coincident with the hot $\mathrm{H_2}$
mass peak. The most intense 0.5 - 10 keV X-ray emission originates
from the nucleus and is oriented north-to-south, similar to the [O
  IV](25.89 $\micron$) emission.  The peak in X-ray emission is
located within the peak in hot $\mathrm{H_2}$ mass suggesting that
X-rays play an important role in exciting the hot $\mathrm{H_2}$
phase.  While there is a correlation between X-ray emission and the
hot $\mathrm{H_2}$ phase, $\mathrm{H_2}$ excitation by X-rays cannot
be distinguished from $\mathrm{H_2}$ excitation by shocks.

%When smoothed to 10$\farcs$2 resolution, the X-ray distribution mimics the [O IV](25.89 $\micron$) emission map.

%; however, due to the disparity in resolution between the X-ray image and the $\mathrm{H_2}$ maps (1$\arcsec$ resolution for X-rays, 10$\farcs$2 resolution for the warm and hot $\mathrm{H_2}$ mass) it is difficult to distinguish between shock excitation and X-ray excitation of $\mathrm{H_2}$ in the nuclear region of NGC 5194.  

We further investigate X-ray excited $\mathrm{H_2}$ emission in M51 by
comparing the 0.5 $-$ 10.0 keV X-ray band to the $\mathrm{H_2}$ S(2)
$-$ $\mathrm{H_2}$ S(5) line intensity maps (Figure \ref{fig13}).  The
$\mathrm{H_2}$ S(2) $-$ $\mathrm{H_2}$ S(5) line intensities all peak
at the X-ray source within the nucleus.  The X-ray source fits within
the $\mathrm{H_2}$ contours because the X-ray image resolution is
slightly higher than the resolution of each of the $\mathrm{H_2}$
maps.

The morphology of the nuclear $\mathrm{H_2}$ emission appears to be
correlated with the X-ray source.  The $\mathrm{H_2}$ S(2) intensity
peaks around the X-ray nucleus and the intensity decreases to the
south by 70 \% in the southern XNC.  The bar structure seen in the
$\mathrm{H_2}$ S(3) emission is aligned with X-ray emission from the
nucleus, XNC, and northern loop.  The $\mathrm{H_2}$ S(3) intensity
peaks between the X-ray nucleus and the XNC.  To the north of the
nucleus, the contours follow the X-ray loop.  To the south of the
nucleus (into the XNC), $\mathrm{H_2}$ S(3) intensity decreases by 80
\%.  The $\mathrm{H_2}$ S(4) and $\mathrm{H_2}$ S(5) line intensities
are highest in the nucleus which coincides with strong X-ray emission
from the nucleus and XNC.

** SAME COMMENT ON SECTION ABOVE.. TOO VERBOSE. 1 PARAGRAPH PLEASE.

\section{Conclusions}

** WE CAN WORK ON THIS LATER.  AGAIN NEED TO MAKE IT CONCISE.

We have spectrally mapped a strip across M51 using the $Spitzer$ IRS
low resolution modules.  We used the spatially resolved spectra to map
$\mathrm{H_2}$ S(0) $-$ $\mathrm{H_2}$ S(5) line intensities across
the strip.  We find:\\ \\ 1.  The morphology of $\mathrm{H_2}$
emission in M51 varies with $\mathrm{H_2}$ rotational level.
$\mathrm{H_2}$ S(0) emission is strongest in the spiral arms of the
galaxy while the higher $J$ transitions show the strongest emission
towards the nucleus.  The $\mathrm{H_2}$ S(1) intensity is strongest
in the nuclear region and in the northwest spiral arms, however, the
peak in $\mathrm{H_2}$ S(0) intensity in the northwest spiral arm is
offset from the peak in $\mathrm{H_2}$ S(0) intensity by 10$\farcs$2.
The $\mathrm{H_2}$ S(2) and $\mathrm{H_2}$ S(3) maps show
$\mathrm{H_2}$ emission in the nucleus, spiral arms, and inter-arm
regions of M51 and bar structure aligned north-to-south is apparent in
$\mathrm{H_2}$ S(3) emission.  $\mathrm{H_2}$ S(4) and $\mathrm{H_2}$
S(5) emission is resolved in the nuclear region of M51.\\
%The $\mathrm{H_2}$ S(5) morphology mimics the $\mathrm{H_2}$ S(3) morphology and the $\mathrm{H_2}$ S(4) morphology shows 
%maps reveal interesting morphology to the molecular gas distributions, such as bar structure across the nucleus in $\mathrm{H_2}$ S(3) emission.\\
\\ 2.  The different morphologies of $\mathrm{H_2}$ emission in M51
indicate significant spatial variations in $\mathrm{H_2}$
excitation-temperature and mass.  Excitation diagrams reveal that the
$\mathrm{H_2}$ exists in a continuous distribution of temperatures
across the galaxy.  Using the low $J$ lines to trace the warm (T = 100
$-$ 300 K) $\mathrm{H_2}$, we find that the warm $\mathrm{H_2}$
excitation-temperature is highest in the nuclear region at 192 K and
the warm $\mathrm{H_2}$ mass peaks in the northwest inner spiral arm
at a mass density of 11 $\mathrm{M_\sun}$/$\mathrm{pc^2}$.  Using the
higher $J$ lines to trace the hot (T = 400 $-$ 1000 K) $\mathrm{H_2}$,
we find that the hot $\mathrm{H_2}$ excitation-temperature is lowest
in the inner spiral arms (500 $-$ 550 K) and increases to $\sim$ 600 K
in the nucleus, where the largest hot $\mathrm{H_2}$ mass densities
are found to be 0.24 $\mathrm{M_\sun}$/$\mathrm{pc^2}$.\\
%We assume that the $\mathrm{H_2}$ exists in the form of two phases, a warm (T = 100 - 300 K) phase and a hot (T = 400 - 1000 K) phase.  
\\ 

3.  The warm and the hot $\mathrm{H_2}$ mass distributions are not
cospatial and the warm-to-hot mass ratio varies across M51.  The hot
mass distribution shows two peaks, one in the nucleus of M51 and one
located interior to the northwest inner spiral arm of M51.  The warm
mass distribution peaks in the northwest spiral arm and is offset from
the hot mass peak by 11$\farcs$4.  The warm-to-hot mass ratio varies
across the galaxy with the ratio being $\sim$ 15 in the nucleus and
increasing to $>$ 100 in the spiral arms.  Variations in the
warm-to-hot $\mathrm{H_2}$ mass ratio and differences in the
morphology of the $\mathrm{H_2}$ emission across M51 indicate that the
primary excitation mechanism differs for the warm and hot
$\mathrm{H_2}$ mass phases as a function of location within the
galaxy.\\ \\ 

4. CO emission is offset from the warm $\mathrm{H_2}$ mass in the
inner spiral arms of M51.  These apparent offsets are real and are
possibly associated with the regions of active star formation within
the molecular clouds.  In the spiral arms, the $\mathrm{H_2}$ S(0) $-$
$\mathrm{H_2}$ S(3) contours trace the CO; however, within the spiral
arms, the peaks in $\mathrm{H_2}$ can be offset from the peaks in CO
intensity.  In the nucleus, the $\mathrm{H_2}$ S(1) $-$ $\mathrm{H_2}$
S(3) lines are brightest and the CO intensity is a factor of $\sim$
2.5 weaker than in the spiral arms suggesting that $\mathrm{H_2}$
emission from the higher $J$ lines is excited by shocks of X-rays.\\
%correlates with the CO emission within the spiral arms.  Within the spiral arms, the $\mathrm{H_2}$ is associated with the CO emission indicating that the $\mathrm{H_2}$ emission is most likely associated with the surface layers of dense PDRs.  However, there are offsets in the location of the peaks in $\mathrm{H_2}$ mass and CO emission.  These offsets are real and are possibly associated with regions that are rich in $\mathrm{H_2}$ but devoid of CO.  \\
\\

5.  Comparing the distributions of $\mathrm{H_2}$ to H$\alpha$ reveals
that the warm and hot $\mathrm{H_2}$ mass is found in the dust lanes
rather than at or around the H$\alpha$ emission regions with the one
exception being that the warm $\mathrm{H_2}$ mass in the inner spiral
arms is coincident with bright H$\alpha$ emission.  This is in
contradiction to previous studies of spatially resolved rovibrational
$\mathrm{H_2}$ line emission in Seyferts that found the $\mathrm{H_2}$
emission to be coincident with the H$\alpha$ emission.\\ \\ 

6.  The peaks in [O IV](25.89 $\micron$) intensity and and X-ray
intensity are both coincident with the peak in hot $\mathrm{H_2}$ mass
in the nucleus of M51 suggesting that the hot $\mathrm{H_2}$ in the
nucleus is primarily excited by the AGN, shocks (possibly associated
with the AGN), or X-rays associated with the AGN.  The spatial
distributions of the [O IV] emission and X-ray surface brightness are
very similar, but a primary excitation mechanism (shocks or X-rays) of
the hot $\mathrm{H_2}$ mass phase cannot be distinguished.  Further
comparison of the $\mathrm{H_2}$ S(2) $-$ $\mathrm{H_2}$ S(5)
intensity maps to the X-ray surface brightness reveal that the nuclear
$\mathrm{H_2}$ emission is associated with X-rays and the bar
structure apparent in the $\mathrm{H_2}$ S(3) map is aligned with the
nucleus, XNC, and northern loop.\\ \\

\acknowledgments

The author graciously acknowledges the Spitzer Science Center Spitzer
Visiting Graduate Student Fellowship program and committee for
providing support for this research.  The author would like to
specifically acknowledge the program coordinators, Phil Appleton and
Alberto Noriega-Crepso.  The author would also like to thank JD Smith
for swift responses to questions about PAHFIT and CUBISM and Nicolas
Flagey for productive discussions about PAHFIT.  Partial support for
the completion and preparation for publication of this study by the
author was provided by AURA grant GO10822.1 to Rice
University.\\ \\ {\it Facilities:} \facility{Spitzer Science Center
  (SSC)}, \facility{Spitzer Space Telescope (SST)},
\facility{Berkely-Illinois-Maryland Array (BIMA)}.

\begin{thebibliography}{}
\bibitem[Aalto et al. 1999]{aal99} Aalto, S., Huttemeister, S., Scoville, N.Z., and Thaddeus, P., 1999, \aj, 522, 165
\bibitem[Abel et al. 2005]{abel05} Abel, N.P., Ferland, G.J., Shaw, G., and van Hoof, P.A.M., 2005, \apjs, 161, 65
\bibitem[Allen et al. 2004]{all04} Allen, R.J., Heaton, H.I., and Kaufman M.J., 2004, \apj, 608, 314
\bibitem[Blitz 1995]{bl95} Blitz, L., 1995, SSR, 684
%\bibitem[Bloeman et al. 1986]{blo86} Bloemen et al., 1986, \aap, 154, 25
\bibitem[Calzetti et al. 2005]{cal05} Calzetti, D. et al., 2005, \apj, 633, 871
\bibitem[Carpenter and Sanders 1998]{car98} Carpenter, J.M. and Sanders, D.B., 1998, \aj, 116, 1856
\bibitem[Crane and van der Hulst 1992]{cra92} Crane, P.C. and van der Hulst, J.M., 1992, \aj, 103, 1146
%\bibitem[Davies et al. 2003]{dav03} Davies, R.I., Sternberg, A., Lehnert, M., and Tacconi-Garman, L.E., 2003, \apj, 597, 907
\bibitem[Dewangan et al. 2005]{dew05} Dewangn, G.C., Griffiths, R.E., Choudhury, M., Miyaji, T., and Schurch, N.J., 2005, \apj, 635, 198
\bibitem[Ford et al. 1985]{ford85} Ford, H.C., Crane, P.C., Jacoby, G.H., Lawrie, D.G., and van der Hulst, J.M., 1985, \apj, 293, 132
%\bibitem[Fuente et al. 1999]{fue99} Fuente et al., 1999, \apj, 518, L45
%\bibitem[Helfer and Blitz 1997]{hel97} Helfer, T.T. and Blitz, L., 1997, 
\bibitem[Helfer et al. 2003]{hel03} Helfer, T.T., Thornley, M.D., Regan, M.W., Wong, T., Sheth, K., Vogel, S.N., Blitz, L., and Bock, D.C.J., 2003, \apjs, 145, 259
\bibitem[Higdon et al. 2006]{hig06} Higdon, S.J.U., Armus, L., Higdon, J.L., Soifer, B.T., and Spoon, H.W.W., 2006, \apj, 648, 323
\bibitem[Houck et al. 2004]{hou04} Houck, J.R. et al., 2004, \apjs, 145, 18
\bibitem[Immler et al. 2002]{imm02} Immler, S., Wilson, A.S., and Terashima, Y., 2002, \apj, 573, L27
\bibitem[Kaufman et al. 2006]{kau06} Kaufman, M.J., Wolfire, M.G., and Hollenbach, D.J., 2006, \apj, 644, 283 
\bibitem[Lord and Young 1990]{lor90} Lord, S.D. and Young, J.S., 1990, \apj, 356, 135
\bibitem[Lutz et al. 1998]{lutz98} Lutz, D., Kunze, D., Spoon, H.W.W., and Thornley, M.D., 1998, \aap, 333, L75
\bibitem[Kennicutt et al. 2003]{ken03} Kennicutt, R.C. et al., 2003, \pasp, 115, 928
\bibitem[Kenny and Lord 1991]{kl91} Kenny, J.D.P. and Lord, S.D., 1991, \apj, 381, 118
\bibitem[Matsushita et a. 2004]{mat04} Matsushita, S. et al., 2004, \apj, 616, L55
%\bibitem[Moorwood and Oliva 1990]{moo90} Moorwood, A.F.M. and Oliva, E., 1990, \aap, 239, 78
%\bibitem[Murphy et al. 2006]{mur06} Murphy, E.J. et al., 2006, \apj, 638, 157
\bibitem[Neufeld et al. 1998]{neu98} Neufeld, D.A., Melnick, G.J., and Harwit, M., 1998, \apj, 506, L75
\bibitem[Neufeld et al. 2006]{neu06} Neufeld, D.A., Melnich, G.J., Sonnentrucker, P., Bergin, E.A., Green, J.D., Kim, K.H., Watson, D.M., Forrest, W.J., and Pipher, J.L., 2006, \apj, 649, 816 
%\bibitem[Pak et al. 2004]{pak04} Pak, S., Jaffe, D.T., Stacey, G.J., Bradford, C>M., Klumpe, E.W., and Keller, L.D. 2004, \apj, 609, 692
\bibitem[Palumbo et al. 1985]{pal85} Palumbo, G.G.C., Fabbiano, G., Fransson, C., and Trinchieri, G., 1985, \apj, 298, 259
\bibitem[Quillen et al. 1999]{qui99} Quillen, A.C., Alonso-Herrero, A., Rieke, M.J., Rieke, G.H., Ruiz, M., Kulkarni, V., 1999, \apj, 527, 696 
%\bibitem[Rand and Kulkarni 1990]{ran90} Rand, R.J. and Kulkarni, S.R., 1990, \apj, 349, L43
\bibitem[Regan et al. 2001]{reg01} Regan, M.W., Thornley, M.D., Helfer, T.T., Sheth, K., Wong, T., Vogel, S.N., Blitz, L., and Bock, D.C.J., 2001, \apj, 561, 218
\bibitem[Rigopoulou et al. 2002]{rig02} Rigopoulou, D., Kunze, D., Lutz, D., Genzel, R., and Moorwood, A.F.M., 2002, \aap, 389, 374
%\bibitem[Rodriguez-Fernandez et al. 2000]{rod00} Rodriguez-Fernandez, Martin-Pintado, de Vicente, Fuente, Huttemeister, Wilson, and Kunze, 2000, \aap, 356, 695
\bibitem[Roussel et al. 2007]{rou07} Roussel, H. et al., 2007, in press
%\bibitem[Rydbeck et al. 1985]{rid85} Ridbeck et al., 1985, \aap, 144, 282
%\bibitem[Sandage and Tammann 1975]{san75} Sandage, A. and Tammann, G.A., 1975, \apj, 196, 313
%\bibitem[Sault et al. 1995]{sau95} Sault et al.,  1995, ASP Conference Series, Vol. 77
\bibitem[Schearer and Stasinska 1999]{ss99} Schaerer, D. and Stasinska, G., 1999, \aap, 345, L17
\bibitem[Scoville and Young 1983]{sco83} Scoville, N.Z. and Young, J.S., 1983, \aj, 265, 148
\bibitem[Scoville et al. 1998]{sco98} Scoville, N.Z., Yun, M.S., Armus, L., and Ford, H., 1998, \apj, 493, L63 
\bibitem[Scoville et al. 2001]{sco01} Scoville, N.Z., Polletta, M., Ewald, S., Stolovy, S.R., Thompson, R., and Rieke, M., 2001, \aj, 122, 3017
\bibitem[Shaw et al. 2005]{shaw05} Shaw, G., Ferland, G.J., Abel, N.P., Stancil, P.C., and van Hoof, P.A.M., 2005, \apj, 624, 794
\bibitem[Sheth et al. 2000]{she00} Sheth, K., Regan, M.W., Vogel, S.N., and Teuben, P.J., 2000, \apj, 532, 221 
\bibitem[Sheth et al. 2002]{she02} Sheth, K., Vogel, S.N., Regan, M.W., Tueben, P.J., Harris, A.I., and Thornley, M.D., 2002, \aj, 124, 2581
\bibitem[Sheth et al. 2005]{she05} Sheth, K., Vogel, S.N., Regan, M.W., Thornley, M.D., and Teuben, P.J., 2005, \apj, 632, 217
\bibitem[DR4]{dr4} SINGS: The Spitzer Infrared Nearby Galaxies Fouth Data Delivery, May 2006.
\bibitem[Smith et al. 2004]{smi04} Smith, J.D.T. et al., 2004, \apjs, 154, 199
%\bibitem[Smith et al. 2006]{smith06} Smith, J.D.T., Armus, L., Buckalew, B., Dale, D.A., Helou, G., Roussel, H., and Sheth, K., 2006, CUBISM User Manual
\bibitem[Smith et al. 2007]{smi07} Smith, J.D.T. et al., 2007, \apj, 656, 770
%\bibitem[Smith et al. 2007]{smith07} Smith, J.D.T. et al., 2007, in press
%\bibitem[Sternberg and Neufeld 1999]{ste99} Sternberg, A. and Neufeld, D.A., 1999, \apj, 516, 371
%\bibitem[Strong et al. 1988]{str88} Strong et al., 1988, \aap, 207, 1
%\bibitem[Takahashi 2001]{tak01}Takahashi, 2001, \apj, 561, 254
\bibitem[Terashima et al. 1998]{ter98} Terashima, Y., Ptak, A., Fujimoto, M.I., Kunieda, H., Makishima, K., and Sherlemitsos, P.J., 1998, \apj, 496, 210
%\bibitem[Terashima and Wilson 2001]{ter01} Terashima, Y. and Wilson, A.S., 2001, \apj, 560, 139
%\bibitem[Timmermann et al. 1996]{tim96} Timmermann et al., 1996, \aap, 315, L281
%\bibitem[Timmermann 1998]{tim98} Timmermann, 1998, \apj, 498, 246
%\bibitem[Valentijn et al. 1996]{val96} Valentijn et al., 1996, \aap, 315, L145
\bibitem[Terashima and Wilson 2001]{wil01} Terashima, Y. and Wilson, A.S., 2001, \apj, 560, 139
%\bibitem[Wright et al. 1996]{wri96} Wright et al., 1996, \aap, 315, L301
\bibitem[Tully 1988]{tul88} Tully, R.B. 1988, Nearby Galaxies catalog (Cambridge:Cambridge University Press)
\bibitem[Vogel et al. 1988]{vog88} Vogel, S.N., Kulkarni, S.R., and Scoville, S.Z., 1988, \nat, 334, 402
\bibitem[Young and Deveraux 1991]{you91} Young, J.S. and Devereux, N.A., 1991, \apj, 373, 414
\bibitem[Young and Scoville 1991]{you91} Young, J.S. and Scoville, N.Z., 1991, \araa, 29, 581
\end{thebibliography}

\clearpage

\begin{figure}
\figurenum{1}
\epsscale{1.1}
\plottwo{f1.eps}{f2.eps}
\plottwo{f3.eps}{f4.eps}
\epsscale{1.1}
\plottwo{f5.eps}{f6.eps}
\caption{Maps of the $\mathrm{H_2}$ S(0) ($top$ $left$), $\mathrm{H_2}$ S(1)
 ($top$ $right$), $\mathrm{H_2}$ S(2) ($middle$ $left$), $\mathrm{H_2}$ S(3) 
 ($middle$ $right$), $\mathrm{H_2}$ S(4) ($bottom$ $left$), and $\mathrm{H_2}$ 
 S(5) ($bottom$ $right$) intensity across the SL and LL strips that we mapped 
 with the Spitzer IRS.  The $\mathrm{H_2}$ S(0) and $\mathrm{H_2}$ S(1) 
 maps are created from the LL data cubes.    
The $\mathrm{H_2}$ S(2), $\mathrm{H_2}$ S(3), $\mathrm{H_2}$ S(4), and 
$\mathrm{H_2}$ S(5) maps are created from the SL data cube.  
The grey-scale is in units of W/$\mathrm{m^2}$.  Contour levels are at 2.9 
$\times$ ${10^{-18}}$, 2.2 $\times$ ${10^{-18}}$, 1.8 $\times$ ${10^{-18}}$, 
1.5 $\times$ ${10^{-18}}$, 1.1 $\times$ ${10^{-18}}$, 7.3 $\times$ ${10^{-19}}$, 
and 3.7 $\times$ ${10^{-19}}$ W/$\mathrm{m^2}$ for $\mathrm{H_2}$ S(0); 9.6 
$\times$ ${10^{-18}}$, 8.6 $\times$ ${10^{-18}}$, 7.5 $\times$ ${10^{-18}}$, 6.4
 $\times$ ${10^{-18}}$, 5.4 $\times$ ${10^{-18}}$, 4.3 $\times$ ${10^{-18}}$, 3.2
  $\times$ ${10^{-18}}$, 2.1 $\times$ ${10^{-18}}$and 1.1 $\times$ ${10^{-18}}$ 
  W/$\mathrm{m^2}$ for $\mathrm{H_2}$ S(1); 1.1 $\times$ ${10^{-18}}$, 8.9 
  $\times$ ${10^{-19}}$, 6.7 $\times$ ${10^{-19}}$, 4.4 $\times$ ${10^{-19}}$, and 
  2.2 $\times$ ${10^{-19}}$ W/$\mathrm{m^2}$ for $\mathrm{H_2}$ S(2); 1.21 
  $\times$ ${10^{-17}}$, 9.4 $\times$ ${10^{-18}}$, 6.7 $\times$ ${10^{-18}}$, 
  4.0 $\times$ ${10^{-18}}$, and 1.3 $\times$ ${10^{-18}}$ W/$\mathrm{m^2}$ 
  for $\mathrm{H_2}$ S(3);  2.0 $\times$ ${10^{-18}}$and 1.0 $\times$ ${10^{-18}}$ 
  W/$\mathrm{m^2}$ for $\mathrm{H_2}$ S(4); 7.3 $\times$ ${10^{-18}}$, 4.0 $\times$ 
  ${10^{-18}}$, and 8.0 $\times$ ${10^{-19}}$ W/$\mathrm{m^2}$ for $\mathrm{H_2}$ S(5).  
  The vertical axis is the right ascension and the horizontal axis is the declination.  
  Note that in all of the maps, north is up and east is to the left.  The different spiral 
  arm regions are labeled on the $\mathrm{H_2}$ S(0) and $\mathrm{H_2}$ S(1) 
  maps in order to aid in discussing the molecular gas morphologies.  The box around 
  the intensity maps represents the SL or LL strip that we mapped.}

%Contours are spaced at 10\%  of the peak flux (6.074 x 10^{-9} W/${m^2}$/sr for $\mathrm{H_2}$S(0) and 1.701 x 10^{-8} W/${m^2}$/sr for $\mathrm{H_2}$S(1)) in each image.  The box around the contour represents the radial strip that was mapped in the IRS LL module.  Note that in all of the maps, north is up and west is to the left.\label{fig1}}

\end{figure}
\clearpage

\clearpage

\begin{figure}
\figurenum{2}
\plottwo{f7.eps}{f8.eps}
\epsscale{.5}
\plotone{f9.eps}
\caption{Excitation diagrams taken from 3 different regions along the M51 strip.  The top two excitation diagrams are taken from regions within the southeast and northwest spiral arms that are 10$\farcs$2 in diameter (1.13 $\times$ $\mathrm{10^5}$ $\mathrm{pc^2}$) and centered at (RA, Dec) of (202.45, 47.21) and (202.49, 47.18), respectively. The excitation diagram at the bottom is taken from from the nuclear region.  The aperture is 10$\farcs$2 in diameter (1.13 $\times$ $\mathrm{10^5}$ $\mathrm{pc^2}$), centered at (RA, Dec) of (202.47, 47.19).}
\end{figure}

\clearpage

\begin{figure}
\epsscale{1.1}
\figurenum{3}
\plottwo{f10.eps}{f11.eps}
\caption{Shown are the warm (T = 100 $-$ 300 K) $\mathrm{H_2}$ ($left$) and hot (T = 400 $-$ 1000 K) $\mathrm{H_2}$ ($right$) mass distributions.  The mass distributions are in units of $\mathrm{M_\sun}$/$\mathrm{pc^2}$.   Contours are overplotted for clarity.  The warm $\mathrm{H_2}$ mass contour levels are at 8.85, 5.55, 4.43, 3.32, 2.21, and 1.10 $\mathrm{M_\sun}$/$\mathrm{pc^2}$.  The hot $\mathrm{H_2}$ contour levels are at 10 \% of 0.25 $\mathrm{M_\sun}$/$\mathrm{pc^2}$.  The hot $\mathrm{H_2}$ mass distribution is derived from the fit to the $\mathrm{H_2}$ S(2) $-$ $\mathrm{H_2}$ S(5) lines and the warm $\mathrm{H_2}$ mass distribution is derived from the fit to the $\mathrm{H_2}$ S(0) $-$ $\mathrm{H_2}$ S(2) lines, corrected for the contribution of the hot $\mathrm{H_2}$ mass phase.}
\end{figure}

\clearpage

\begin{figure}
\figurenum{4}
\epsscale{.55}
\plotone{f12.eps}
\caption{The warm (T = 100 $-$ 300 K) $\mathrm{H_2}$ mass distribution compared 
to the warm $\mathrm{H_2}$ excitation-temperature.  The warm $\mathrm{H_2}$ 
excitation-temperature and mass distributions are derived from the fit to the excitation 
diagrams across the strip for the $\mathrm{H_2}$ S(0) $-$ $\mathrm{H_2}$ S(2) lines, 
corrected for the contribution of the hot (T = 400 $-$ 1000 K) $\mathrm{H_2}$ phase.  
Mass density contour levels are at 8.85, 5.55, 4.43, 3.32, 2.21, and 1.10 
$\mathrm{M_\sun}$/$\mathrm{pc^2}$ (same as in Figure \ref{fig3}). 
The grey-scale represents the excitation-temperature distribution (in units of Kelvin).  
The non-rectangular shape to the map is due to the slight offset of the $Spitzer$ IRS SL 
strip relative to the LL strip.}
\end{figure}

\clearpage
\begin{figure}
\figurenum{5}
\epsscale{.55}
\plotone{f13.eps}
\caption{The hot (T = 400 $-$ 1000 K) $\mathrm{H_2}$ mass distribution compared 
to the hot $\mathrm{H_2}$ excitation-temperature.  The hot $\mathrm{H_2}$ excitation-temperature 
and mass distributions are derived from the fit to the excitation diagrams across the 
strip for the $\mathrm{H_2}$ S(2) $-$ $\mathrm{H_2}$ S(5) lines.  Mass density 
contour levels are at 10\% of 0.25 $\mathrm{M_\sun}$/$\mathrm{pc^2}$ (same as in Figure \ref{fig3}). 
The grey-scale represents the excitation-temperature distribution (in units of Kelvin).}
\end{figure}

\clearpage

\begin{figure}
\epsscale{.55}
\figurenum{6}
\plotone{f14.eps}
\caption{The warm (T = 100 $-$ 300 K) $\mathrm{H_2}$ mass (in grey-scale) compared to
 the hot (T = 400 $-$ 1000 K) $\mathrm{H_2}$ mass (in contours).  Contours levels for the 
 hot $\mathrm{H_2}$ mass distribution are at 10\% of the maximum mass density 
 0.25 $\mathrm{M_\sun}$/$\mathrm{pc^2}$.  The grey-scale is in units of 
 $\mathrm{M_\sun}$/$\mathrm{pc^2}$.}
\end{figure}

\clearpage

\begin{figure}
\epsscale{1.1}
\figurenum{7}
\plottwo{f15.eps}{f16.eps}
\caption{$Left$: Comparison of  CO intensity (in grey-scale) to the warm (T = 100 $-$ 300 K) 
$\mathrm{H_2}$ mass (in contours).  The CO intensity is in units of Jy beam $\mathrm{s^{-1}}$. 
The warm $\mathrm{H_2}$ mass contours are the same as in Figures \ref{fig3} and \ref{fig4}.  
$Right$: Comparison of CO intensity (in grey-scale) to the hot (T = 400 $-$ 1000 K) 
$\mathrm{H_2}$ mass (in contours).  The CO intensity is in units of Jy beam 
$\mathrm{s^{-1}}$. The hot $\mathrm{H_2}$ mass contours are the same as in 
Figures \ref{fig3} and \ref{fig5}.}
\end{figure}

\clearpage

\begin{figure}
\epsscale{1.1}
\figurenum{8}
\plottwo{f17.eps}{f18.eps}
\plottwo{f19.eps}{f20.eps}
\caption{Comparison of the CO emission to the $\mathrm{H_2}$ S(0) ($top$ $left$), 
 $\mathrm{H_2}$ S(1) ($top$ $right$),  $\mathrm{H_2}$ S(2) ($bottom$ $left$),  and 
 $\mathrm{H_2}$ S(3) ($bottom$ $right$) emission.  The CO emission maps are in 
 units of Jy beam $\mathrm{s^{-1}}$.  Contour levels for $\mathrm{H_2}$ S(0), 
 $\mathrm{H_2}$ S(1), $\mathrm{H_2}$ S(2), and $\mathrm{H_2}$ S(3) are the 
 same as in Figure \ref{fig1}.}
\end{figure}

\clearpage

\begin{figure}
\epsscale{1.1}
\figurenum{9}
\plottwo{f21.eps}{f22.eps}
\caption{$Left$: Comparison of  H$\alpha$ (in grey-scale) to the warm (T = 100 $-$ 300 K) 
$\mathrm{H_2}$ mass (in contours).  The H$\alpha$ image is in units of counts/sec. The 
warm $\mathrm{H_2}$ mass contours are the same as in Figures \ref{fig3} and \ref{fig4}.  
$Right$: Comparison of H$\alpha$ (in grey-scale) to the hot (T = 400 $-$ 1000 K) $\mathrm{H_2}$ 
mass (in contours).  The H$\alpha$ image is in units of counts/sec. The hot $\mathrm{H_2}$ 
mass contours are the same as in Figures \ref{fig3} and \ref{fig5}.}
\end{figure}

\clearpage

\begin{figure}
\epsscale{1.1}
\figurenum{10}
\plottwo{f23.eps}{f24.eps}
\plottwo{f25.eps}{f26.eps}
\caption{Comparison of H$\alpha$ emission to the $\mathrm{H_2}$ S(0) ($top$ $left$),  
$\mathrm{H_2}$ S(1) ($top$ $right$),  $\mathrm{H_2}$ S(2) ($bottom$ $left$),  and 
$\mathrm{H_2}$ S(3) ($bottom$ $right$) emission.  The H$\alpha$ image is in units 
of counts/s.  Contour levels for $\mathrm{H_2}$ S(0), $\mathrm{H_2}$ S(1), 
$\mathrm{H_2}$ S(2), and $\mathrm{H_2}$ S(3) are the same as in Figure \ref{fig1}.}
\end{figure}

\clearpage

\begin{figure}
\epsscale{1.1}
\figurenum{11}
\plottwo{f27.eps}{f28.eps}
\caption{$Left$:  Comparison of the [O IV](25.89 \micron) emission (in grey-scale) to 
the warm (T = 100 K - 300 K) $\mathrm{H_2}$ mass distribution (in contours).  
Hot $\mathrm{H_2}$ mass contours are the same as in Figures \ref{fig3} and \ref{fig4}.  
$Right$: Comparison of the [O IV](25.89 \micron) emission (in grey-scale) to the hot 
(T = 400 - 1000 K) $\mathrm{H_2}$ mass distribution (in contours).  Hot $\mathrm{H_2}$ 
mass contours are the same as in Figures \ref{fig3} and \ref{fig5}.  The [O IV](25.89 \micron)
 emission is in units of W/$\mathrm{m^2}$.}
\end{figure}

\clearpage

\begin{figure}
\epsscale{1.1}
\figurenum{12}
\plottwo{f29.eps}{f30.eps}
\caption{$Left$:  Comparison of the smoothed 0.5 $-$ 10 keV X-ray emission band 
(in grey-scale) to the warm (T = 100 $-$ 300 K) $\mathrm{H_2}$ mass distribution 
(in contours).   The X-ray image has been smoothed to the same resolution as the 
warm $\mathrm{H_2}$ mass map.  X-ray emission is in units of counts.  $\mathrm{H_2}$ 
mass contours are the same as in Figures \ref{fig3} and \ref{fig4}.  $Right$: Comparison 
of the smoothed 0.5 $-$ 10 keV X-ray emission band (in grey-scale) to the hot 
(T = 400 $-$ 1000 K) $\mathrm{H_2}$ mass distribution (in contours).  The 
$\mathrm{H_2}$ mass distribution contours are the same as in Figures \ref{fig3} 
and \ref{fig5}.}
\end{figure}

\clearpage

\begin{figure}[!h]
\epsscale{1.1}
\figurenum{13}
\plottwo{f31.eps}{f32.eps}
\plottwo{f33.eps}{f34.eps}
\caption{Comparison of the 0.5 $-$ 10 keV X-ray emission band (in grey-scale) to the 
$\mathrm{H_2}$ S(2) ($top$ $left$), $\mathrm{H_2}$ S(3) ($top$ $right$), 
$\mathrm{H_2}$ S(4) ($bottom$ $left$), and $\mathrm{H_2}$ S(5) ($bottom$ $right$) 
emission in the nuclear region of M51.  X-ray emission is in units of counts.  
The $\mathrm{H_2}$ S(2) and $\mathrm{H_2}$ S(3) emission contours are at 
10 \% of their peak values (2.20 $\times$ ${10^{-18}}$ and 1.35 $\times$ ${10^{-17}}$ 
W/$\mathrm{m^2}$, respectively).  The $\mathrm{H_2}$ S(4) contours are at  
2.0 $\times$ ${10^{-18}}$ and 1.0 $\times$ ${10^{-18}}$ W/$\mathrm{m^2}$ and 
the $\mathrm{H_2}$ S(5) contours are at 7.3 $\times$ ${10^{-18}}$, 4.0 
$\times$ ${10^{-18}}$, and 8.0 $\times$ ${10^{-19}}$ W/$\mathrm{m^2}$.}
\end{figure}


\clearpage

%\begin{deluxetable}{cccccc}
%\tabletypesize{\scriptsize}
%\rotate
%\tablecaption{Resolution of the $\mathrm{H_2}$ Maps\label{tbl-2}}
%\tablewidth{0pt}
%\tablehead{
%\colhead{Transition} & \colhead{Wavelength (\micron)} & \colhead{Spatial Resolution} & \colhead{\lambda/\delta\lambda}
%}
%\startdata
%$\mathrm{H_2}$(0-0)S(0) & 28.22 & 10$\farcs$2 & 155  \\
%$\mathrm{H_2}$(0-0)S(1) & 17.04 & 6$\farcs$17 & 185 \\
%$\mathrm{H_2}$(0-0)S(2) & 12.28 & 4$\farcs$37 & 198 \\
%$\mathrm{H_2}$(0-0)S(3) & 9.66 & 3$\farcs$44 & 156 \\
%$\mathrm{H_2}$(0-0)S(4) & 8.03 & 2$\farcs$85 & 129 \\
%$\mathrm{H_2}$(0-0)S(5) & 6.91 & 2$\farcs$46 & 223 \\
%\enddata
%\tablecomments{Table lists the resolution of the $\mathrm{H_2}$ lines for which the extinction corrected flux has been mapped across the strip over \objectname{NGC 5194}.} 
%\end{deluxetable}


%\clearpage


\begin{deluxetable}{cccccc}
\tabletypesize{\scriptsize}
\rotate
\tablecaption{$\mathrm{H_2}$ Parameters %and Resolution of the $\mathrm{H_2}$ Maps
\label{tbl-1}}
\tablewidth{0pt}
\tablehead{
\colhead{Transition} & \colhead{Wavelength (\micron)} & \colhead{Rotational State (J)} & \colhead{Energy (E/k)} & \colhead{A ($\mathrm{s^{-1}})} & \colhead{Statistical Weight (g)} %& \colhead{Spatial Resolution} & \colhead{\lambda/\delta\lambda}
}
\startdata
$\mathrm{H_2}$(0-0)S(0) & 28.22 & 2 & 510 & 2.94$\times$${10^{-11}}$ & 5 \\ %& 10$\farcs$2 & 155 \\
$\mathrm{H_2}$(0-0)S(1) & 17.04 & 3 & 1015 & 4.76$\times$${10^{-10}}$ & 21 \\ % & 6$\farcs$17 & 185 \\
$\mathrm{H_2}$(0-0)S(2) & 12.28 & 4 & 1682 & 2.76$\times$${10^{-9}}$ & 9 \\ %& 4$\farcs$37 & 198 \\
$\mathrm{H_2}$(0-0)S(3) & 9.66 & 5 & 2504 & 9.84$\times$${10^{-9}}$ & 33 \\ %& 3$\farcs$44 & 156 \\
$\mathrm{H_2}$(0-0)S(4) & 8.03 & 6 & 3474 & 2.64$\times$${10^{-8}}$ & 13 \\ %& 2$\farcs$85 & 129 \\
$\mathrm{H_2}$(0-0)S(5) & 6.91 & 7 & 4586 & 5.88$\times$${10^{-8}}$ & 45 \\ % & 2$\farcs$46 & 223 \\
%$\mathrm{H_2}$(0-0)S(6) & 6.11 & 8 & 5829 & 1.14$\times$${10^{-7}}$ & 17 \\
%$\mathrm{H_2}$(0-0)S(7) & 5.51 & 9 & 7197 & 2.00$\times$${10^{-7}}$ & 57 \\
\enddata
\tablecomments{The statistical weight (g) is (2$J$ +1)(2$I$+1) where $I$ equals 1 for odd J transitions (ortho transitions) and $I$ equals 0 for even J transitions (para transitions).} 
\end{deluxetable}



\end{document}

