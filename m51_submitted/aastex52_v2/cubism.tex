

\documentclass[manuscript]{aastex}


%\shorttitle{Collapsed Cores in Globular Clusters}
%\shortauthors{Djorgovski et al.}

\begin{document}

%\title{Spectra Reduced and Extracted in CUBISM}

%\author{Gregory J. Brunner}
%\affil{Rice University}



%\begin{abstract}

%\end{abstract}


\section{Reduction in CUBISM}

We reduced the 25 HII regions in CUBISM (Kennicut et al. 2003, Smith et al. 2004, Smith et al. 2007).
%\ref{ken03, smi04, smi07}.  
The advantage of using CUBISM is that it allowed us to define the  
2-D spatial region over which we wanted to extract spectra, thus allowing us 
to increase the signal-to-noise of the spectra by extracting over the 
brightest part of the H II regions.  For each H II region, we constructed 
a data cube.  Global bad pixels were removed manually; record level 
bad pixels that deviated by 5$\sigma$ were removed automatically 
within CUBISM.  In CUBISM, the native pixel size is 2.26 arcsec pixel$^{-1}$ 
for spectra taken in the SH module.  We varied our extraction aperture 
size due to the differences in size of the grid over the HII regions.  Table 1
%\ref{table1} 
lists the HII region and the extraction aperture size.  

%Spectra 
%in CUBISM are output in units of MJy sr$^{-1}$.  In order to convert the 
%spectra from MJy sr$^{-1}$, we multiplied the spectra by a factor of 
%$aperture$ $size$ (in pixel$^2$) $\times$ (3.57 $\times$ 10$^{-5}$) \times 
%(\lambda/9.9718)$^2$.  Doing a straightforward units conversion, our 
%multiplicative factor for converting from MJy sr$^{-1}$ to Jy should be 
%$aperture$ $size$ (in pixel$^2$) $\times$ (2.26 arcsec/pixel)$^2$ 
%$\times$ 2.35  $\times$ 10$^{-11}$ (sr/arcsec$^2$) \times 10$^6$ (Jy/MJy) = 
%$aperture$ $size$ $\times$ 1.20 $\times$ 10$^{-4}$.










\begin{thebibliography}{}
\bibitem[Kennicutt et al. 2003]{ken03} Kennicutt, R.C. et al., 2003, \pasp, 115, 928
\bibitem[Smith et al. 2004]{smi04} Smith, J.D.T. et al., 2004, \apjs, 154, 199
\bibitem[Smith et al. 2007]{smi07} Smith, J.D.T. et al., 2007, \pasp, submitted
\end{thebibliography}

\clearpage

\begin{deluxetable}{cc}
\tablewidth{0pt}
\label{table1}
\tablecaption{CUBISM Extraction Aperture Sizes}
\tablehead{
\colhead{H II Region}           & \colhead{Aperture size (pixel$^2$)}}
\startdata
4 & 6 x 6\\
27 & 6 x 4\\
32 & 6 x 6\\
33 & 6 x 6\\
42 & 6 x 4\\
45 & 10 x 6\\
62 & 6 x 6\\
79 & 6 x 6\\
87e & 6 x 6\\
88w & 6 x 6\\
95 & 6 x 6\\
214 & 6 x 6\\
230 & 6 x 6\\
251 & 6 x 6\\
277 & 10 x 6\\
280 & 9 x 8\\
301 & 6 x 6\\
302 & 6 x 6\\
623 & 6 x 6\\
638 & 9 x 6\\
651 & 6 x 6\\
691 & 6 x 6\\
702 & 6 x 4\\
710 & 6 x 6\\
740w & 6 x 6\\
\enddata
%}
\end{deluxetable}

\end{document}

